\documentclass[]{book}
\usepackage{lmodern}
\usepackage{amssymb,amsmath}
\usepackage{ifxetex,ifluatex}
\usepackage{fixltx2e} % provides \textsubscript
\ifnum 0\ifxetex 1\fi\ifluatex 1\fi=0 % if pdftex
  \usepackage[T1]{fontenc}
  \usepackage[utf8]{inputenc}
\else % if luatex or xelatex
  \ifxetex
    \usepackage{mathspec}
  \else
    \usepackage{fontspec}
  \fi
  \defaultfontfeatures{Ligatures=TeX,Scale=MatchLowercase}
\fi
% use upquote if available, for straight quotes in verbatim environments
\IfFileExists{upquote.sty}{\usepackage{upquote}}{}
% use microtype if available
\IfFileExists{microtype.sty}{%
\usepackage{microtype}
\UseMicrotypeSet[protrusion]{basicmath} % disable protrusion for tt fonts
}{}
\usepackage{hyperref}
\hypersetup{unicode=true,
            pdftitle={PostgreSQL Explained for R-Users and R-Programmers},
            pdfauthor={Ben Gonzalez},
            pdfborder={0 0 0},
            breaklinks=true}
\urlstyle{same}  % don't use monospace font for urls
\usepackage{natbib}
\bibliographystyle{apalike}
\usepackage{color}
\usepackage{fancyvrb}
\newcommand{\VerbBar}{|}
\newcommand{\VERB}{\Verb[commandchars=\\\{\}]}
\DefineVerbatimEnvironment{Highlighting}{Verbatim}{commandchars=\\\{\}}
% Add ',fontsize=\small' for more characters per line
\usepackage{framed}
\definecolor{shadecolor}{RGB}{248,248,248}
\newenvironment{Shaded}{\begin{snugshade}}{\end{snugshade}}
\newcommand{\KeywordTok}[1]{\textcolor[rgb]{0.13,0.29,0.53}{\textbf{#1}}}
\newcommand{\DataTypeTok}[1]{\textcolor[rgb]{0.13,0.29,0.53}{#1}}
\newcommand{\DecValTok}[1]{\textcolor[rgb]{0.00,0.00,0.81}{#1}}
\newcommand{\BaseNTok}[1]{\textcolor[rgb]{0.00,0.00,0.81}{#1}}
\newcommand{\FloatTok}[1]{\textcolor[rgb]{0.00,0.00,0.81}{#1}}
\newcommand{\ConstantTok}[1]{\textcolor[rgb]{0.00,0.00,0.00}{#1}}
\newcommand{\CharTok}[1]{\textcolor[rgb]{0.31,0.60,0.02}{#1}}
\newcommand{\SpecialCharTok}[1]{\textcolor[rgb]{0.00,0.00,0.00}{#1}}
\newcommand{\StringTok}[1]{\textcolor[rgb]{0.31,0.60,0.02}{#1}}
\newcommand{\VerbatimStringTok}[1]{\textcolor[rgb]{0.31,0.60,0.02}{#1}}
\newcommand{\SpecialStringTok}[1]{\textcolor[rgb]{0.31,0.60,0.02}{#1}}
\newcommand{\ImportTok}[1]{#1}
\newcommand{\CommentTok}[1]{\textcolor[rgb]{0.56,0.35,0.01}{\textit{#1}}}
\newcommand{\DocumentationTok}[1]{\textcolor[rgb]{0.56,0.35,0.01}{\textbf{\textit{#1}}}}
\newcommand{\AnnotationTok}[1]{\textcolor[rgb]{0.56,0.35,0.01}{\textbf{\textit{#1}}}}
\newcommand{\CommentVarTok}[1]{\textcolor[rgb]{0.56,0.35,0.01}{\textbf{\textit{#1}}}}
\newcommand{\OtherTok}[1]{\textcolor[rgb]{0.56,0.35,0.01}{#1}}
\newcommand{\FunctionTok}[1]{\textcolor[rgb]{0.00,0.00,0.00}{#1}}
\newcommand{\VariableTok}[1]{\textcolor[rgb]{0.00,0.00,0.00}{#1}}
\newcommand{\ControlFlowTok}[1]{\textcolor[rgb]{0.13,0.29,0.53}{\textbf{#1}}}
\newcommand{\OperatorTok}[1]{\textcolor[rgb]{0.81,0.36,0.00}{\textbf{#1}}}
\newcommand{\BuiltInTok}[1]{#1}
\newcommand{\ExtensionTok}[1]{#1}
\newcommand{\PreprocessorTok}[1]{\textcolor[rgb]{0.56,0.35,0.01}{\textit{#1}}}
\newcommand{\AttributeTok}[1]{\textcolor[rgb]{0.77,0.63,0.00}{#1}}
\newcommand{\RegionMarkerTok}[1]{#1}
\newcommand{\InformationTok}[1]{\textcolor[rgb]{0.56,0.35,0.01}{\textbf{\textit{#1}}}}
\newcommand{\WarningTok}[1]{\textcolor[rgb]{0.56,0.35,0.01}{\textbf{\textit{#1}}}}
\newcommand{\AlertTok}[1]{\textcolor[rgb]{0.94,0.16,0.16}{#1}}
\newcommand{\ErrorTok}[1]{\textcolor[rgb]{0.64,0.00,0.00}{\textbf{#1}}}
\newcommand{\NormalTok}[1]{#1}
\usepackage{longtable,booktabs}
\usepackage{graphicx,grffile}
\makeatletter
\def\maxwidth{\ifdim\Gin@nat@width>\linewidth\linewidth\else\Gin@nat@width\fi}
\def\maxheight{\ifdim\Gin@nat@height>\textheight\textheight\else\Gin@nat@height\fi}
\makeatother
% Scale images if necessary, so that they will not overflow the page
% margins by default, and it is still possible to overwrite the defaults
% using explicit options in \includegraphics[width, height, ...]{}
\setkeys{Gin}{width=\maxwidth,height=\maxheight,keepaspectratio}
\IfFileExists{parskip.sty}{%
\usepackage{parskip}
}{% else
\setlength{\parindent}{0pt}
\setlength{\parskip}{6pt plus 2pt minus 1pt}
}
\setlength{\emergencystretch}{3em}  % prevent overfull lines
\providecommand{\tightlist}{%
  \setlength{\itemsep}{0pt}\setlength{\parskip}{0pt}}
\setcounter{secnumdepth}{5}
% Redefines (sub)paragraphs to behave more like sections
\ifx\paragraph\undefined\else
\let\oldparagraph\paragraph
\renewcommand{\paragraph}[1]{\oldparagraph{#1}\mbox{}}
\fi
\ifx\subparagraph\undefined\else
\let\oldsubparagraph\subparagraph
\renewcommand{\subparagraph}[1]{\oldsubparagraph{#1}\mbox{}}
\fi

%%% Use protect on footnotes to avoid problems with footnotes in titles
\let\rmarkdownfootnote\footnote%
\def\footnote{\protect\rmarkdownfootnote}

%%% Change title format to be more compact
\usepackage{titling}

% Create subtitle command for use in maketitle
\providecommand{\subtitle}[1]{
  \posttitle{
    \begin{center}\large#1\end{center}
    }
}

\setlength{\droptitle}{-2em}

  \title{PostgreSQL Explained for R-Users and R-Programmers}
    \pretitle{\vspace{\droptitle}\centering\huge}
  \posttitle{\par}
    \author{Ben Gonzalez}
    \preauthor{\centering\large\emph}
  \postauthor{\par}
      \predate{\centering\large\emph}
  \postdate{\par}
    \date{2019-10-11}

\usepackage{booktabs}

\begin{document}
\maketitle

{
\setcounter{tocdepth}{1}
\tableofcontents
}
\chapter{Prerequisites}\label{prerequisites}

For anyone interested in using this book you will need the following
packages and tools to follow along.

\begin{itemize}
\tightlist
\item
  RPostgresql
\item
  DBI
\item
  Your own PostgreSQL database
\item
  Remote access to your database
\end{itemize}

The \textbf{RPostgreSQL and DBI} package can be installed from CRAN or
Github:

\begin{Shaded}
\begin{Highlighting}[]
\KeywordTok{install.packages}\NormalTok{(}\StringTok{"RPostgreSQL"}\NormalTok{)}
\KeywordTok{install.packages}\NormalTok{(}\StringTok{"DBI"}\NormalTok{)}
\NormalTok{devtools}\OperatorTok{::}\KeywordTok{install_git}\NormalTok{(}\StringTok{'https://github.com/r-dbi/DBI.git'}\NormalTok{)}
\NormalTok{devtools}\OperatorTok{::}\KeywordTok{install_git}\NormalTok{(}\StringTok{'https://github.com/cran/RPostgreSQL.git)}
\end{Highlighting}
\end{Shaded}

Notes on using SQL syntax in RPostgreSQL

To successfully query data in PostgreSQL the following caveats may be
necessary. This is especially the case if someone has created column
names that are unique and odd in some form or fashion.

\begin{itemize}
\tightlist
\item
  use backslahes ~to escape the quotes ``'' that are necessary when
  querying data.
\item
  use quotes ``'' around camel back title cases e.g. \textbf{``Medu''}
  or \textbf{``Fedu''}
\end{itemize}

\chapter{Example data to follow along
with}\label{example-data-to-follow-along-with}

You can also download the example datasets and place them in your own
PostgreSQL database.

Detroit Dataset: 14 Columns with 13 Observations

This is the data set called DETROIT' in the bookSubset selection in
regression' by Alan J. Miller published in the Chapman \& Hall series of
monographs on Statistics \& Applied Probability, no. 40. The data are
unusual in that a subset of three predictors can be found which gives a
very much better fit to the data than the subsets found from the
Efroymson stepwise algorithm, or from forward selection or backward
elimination. The original data were given in appendix A of `Regression
analysis and its application: A data-oriented approach' by Gunst \&
Mason, Statistics textbooks and monographs no. 24, Marcel Dekker. It has
caused problems because some copies of the Gunst \& Mason book do not
contain all of the data, and because Miller does not say which variables
he used as predictors and which is the dependent variable. (HOM was the
dependent variable, and the predictors were FTP \ldots{} WE)

Source: \url{http://lib.stat.cmu.edu/datasets/detroit}

\begin{center}\rule{0.5\linewidth}{\linethickness}\end{center}

Create Detroit Table in PostgreSQL

The necessary files to create the Detroit table in PostgreSQL can be
found in the following link

Download Detroit data

\begin{center}\rule{0.5\linewidth}{\linethickness}\end{center}

Student Performance Data Set: A data frame with 392 rows and 33
variables:

This data approach looks at student achievement in secondary education
of two Portuguese schools. The data attributes include student grades,
demographics, social and school related features and it was collected
utilizing school reports and questionnaires. Two datasets are provided
regarding the performance in two distinct subjects: Mathematics (mat)
and Portuguese language (por). In {[}Cortez and Silva, 2008{]}, the two
datasets were modeled under binary/five-level classification and
regression tasks. Important note: the target attribute G3 has a strong
correlation with attributes G2 and G1. This occurs because G3 is the
final year grade (issued at the 3rd period), While G1 and G2 correspond
to the 1st and 2nd period grades. It is more difficult to predict G3
without G2 and G1, but such prediction is much more useful (see paper
source for more details)

Source: \url{http://archive.ics.uci.edu/ml/datasets/Student+Performance}

\begin{center}\rule{0.5\linewidth}{\linethickness}\end{center}

Create Student Table in PostgreSQL

The necessary files to create the Student table in PostgreSQL can be
found in the following link

Download Student data

\chapter{Introduction}\label{introduction}

This is a reference book on how to user PostgreSQL in Rstudio utilizing
the DBI and RPostgresql packages.

Note: All queries are limited to ten rows to allow for easier reading
and understanding.

After searching the internet for exstensive books I was unable to find
anything to my liking. Working with databases is key to get things done
in R, Rstudio, Python, and R-Shiny. So I wanted to write my own book
aimed at practical knowledge on how to do things and hopefully create a
good work out of the hodgepodge of junk that is out there.

First things first. You will want to ensure that you have enabled remote
access to your PostgreSQL database.

How to allow remote access to PostgreSQL database:

You will need to change some configurations in the postgresql.conf file
on your server.

\begin{Shaded}
\begin{Highlighting}[]
\FunctionTok{find} \DataTypeTok{\textbackslash{} }\NormalTok{-name }\StringTok{"postgresql.conf"}

\FunctionTok{sudo}\NormalTok{ nano /var/lib/pgsql/PSQLVERSION/data/postgresql.conf}
\end{Highlighting}
\end{Shaded}

The you will want to change the following line listen\_addresses =
`localhost' to listen\_addresses = `*':

Search for it using CTRL + W

\begin{Shaded}
\begin{Highlighting}[]
\ExtensionTok{listen_addresses}\NormalTok{ = }\StringTok{'localhost'}
\end{Highlighting}
\end{Shaded}

\begin{Shaded}
\begin{Highlighting}[]
\ExtensionTok{listen_addresses}\NormalTok{ = }\StringTok{'*'}
\end{Highlighting}
\end{Shaded}

Next restart your PostgreSQL database.

\begin{Shaded}
\begin{Highlighting}[]
\FunctionTok{sudo}\NormalTok{ systemctl postgresql restart}
\end{Highlighting}
\end{Shaded}

You should still receive an error as you also need to configure the
pg\_hba.conf file as well.

\begin{Shaded}
\begin{Highlighting}[]
\FunctionTok{find} \DataTypeTok{\textbackslash{} }\NormalTok{-name }\StringTok{"pg_hba.conf"}

\FunctionTok{sudo}\NormalTok{ nano /var/lib/pgsql/PSQLVERSION/data/pg_hba.conf}
\end{Highlighting}
\end{Shaded}

Now place the followin at the very end of the file.

\begin{Shaded}
\begin{Highlighting}[]
\ExtensionTok{host}\NormalTok{    all             all              0.0.0.0/0                       md5}
\ExtensionTok{host}\NormalTok{    all             all              ::/0                            md5}
\end{Highlighting}
\end{Shaded}

\url{https://blog.bigbinary.com/2016/01/23/configure-postgresql-to-allow-remote-connection.html}

Step 1: Install the necessary packages to check the connection.

\begin{Shaded}
\begin{Highlighting}[]
\KeywordTok{library}\NormalTok{(RPostgreSQL)}
\KeywordTok{library}\NormalTok{(DBI)}
\end{Highlighting}
\end{Shaded}

Step 2: We want to connect to our PostgreSQL database itself. I
recommend utilizing Digital Ocean to host your own cloud base PostgreSQL
instance. Here is a link to a tutorial on their website to build your
own if you have not done so before. Digital Ocean PostgreSQL. In the
below code chunk you will want to update the repsective values with the
values from your database instance.

\begin{Shaded}
\begin{Highlighting}[]
\KeywordTok{library}\NormalTok{(DBI)}
\KeywordTok{library}\NormalTok{(RPostgreSQL)}
\NormalTok{DBI}\OperatorTok{::}\KeywordTok{dbDriver}\NormalTok{(}\StringTok{'PostgreSQL'}\NormalTok{)}
\KeywordTok{require}\NormalTok{(RPostgreSQL)}
\NormalTok{drv=}\KeywordTok{dbDriver}\NormalTok{(}\StringTok{"PostgreSQL"}\NormalTok{)}
\NormalTok{con=}\KeywordTok{dbConnect}\NormalTok{(drv,}\DataTypeTok{dbname=}\NormalTok{dbname,}\DataTypeTok{host=}\NormalTok{dbhost,}\DataTypeTok{port=}\DecValTok{5432}\NormalTok{,}\DataTypeTok{user=}\NormalTok{dbuser,}\DataTypeTok{password=}\NormalTok{dbpassword)}
\end{Highlighting}
\end{Shaded}

\begin{itemize}
\tightlist
\item
  \textbf{db} name will be the database you are wanting to use.
\item
  \textbf{host} will be the host your database is on. Either your
  localhost or the url to your database.
\item
  \textbf{port} By default the port will be 5432 for postgresql.
\item
  \textbf{user} will be the username for the database you are connecting
  to
\item
  \textbf{password} will be the database password you use when
  connecting to postgresql Next we can list the tables that are
  available in our database
\end{itemize}

This is the DBI way to do it in Rstudio.

\begin{Shaded}
\begin{Highlighting}[]
\KeywordTok{dbListTables}\NormalTok{(}\DataTypeTok{conn =}\NormalTok{ con)}
\end{Highlighting}
\end{Shaded}

\begin{verbatim}
## [1] "teachers" "dodgers"  "student"  "detroit"
\end{verbatim}

This is the SQL syntax way to do it. Here we can see the
tablename,tableowner, and the tablespace along with other housekeeping
items that may be of interest to us.

\begin{Shaded}
\begin{Highlighting}[]
\KeywordTok{SELECT}\NormalTok{ * }\KeywordTok{FROM}\NormalTok{ pg_catalog.pg_tables;}
\end{Highlighting}
\end{Shaded}

\label{tab:unnamed-chunk-8}Displaying records 1 - 10

schemaname

tablename

tableowner

tablespace

hasindexes

hasrules

hastriggers

rowsecurity

public

teachers

ben

NA

FALSE

FALSE

FALSE

FALSE

pg\_catalog

pg\_statistic

postgres

NA

TRUE

FALSE

FALSE

FALSE

pg\_catalog

pg\_type

postgres

NA

TRUE

FALSE

FALSE

FALSE

public

dodgers

ben

NA

FALSE

FALSE

FALSE

FALSE

public

student

ben

NA

FALSE

FALSE

FALSE

FALSE

pg\_catalog

pg\_policy

postgres

NA

TRUE

FALSE

FALSE

FALSE

pg\_catalog

pg\_authid

postgres

pg\_global

TRUE

FALSE

FALSE

FALSE

public

detroit

ben

NA

FALSE

FALSE

FALSE

FALSE

pg\_catalog

pg\_user\_mapping

postgres

NA

TRUE

FALSE

FALSE

FALSE

pg\_catalog

pg\_subscription

postgres

pg\_global

TRUE

FALSE

FALSE

FALSE

Select * FROM table;

Here we are querying the entire table and bringing back all of the
values.

\textbf{Teachers Dataset}

\begin{Shaded}
\begin{Highlighting}[]
\KeywordTok{SELECT}\NormalTok{ * }\KeywordTok{FROM}\NormalTok{ teachers}
\end{Highlighting}
\end{Shaded}

\label{tab:unnamed-chunk-9}6 records

id

first\_name

last\_name

school

hire\_date

salary

1

Janet

Smith

F.D. Roosevelt HS

2011-10-30

36200

2

Lee

Reynolds

F.D. Roosevelt HS

1993-05-22

65000

3

Samuel

Cole

Myers Middle School

2005-08-01

43500

4

Samantha

Bush

Myers Middle School

2011-10-30

36200

5

Betty

Diaz

Myers Middle School

2005-08-30

43500

6

Kathleen

Roush

F.D. Roosevelt HS

2010-10-22

38500

\textbf{Detroit Dataset}

\begin{Shaded}
\begin{Highlighting}[]
\KeywordTok{SELECT}\NormalTok{ * }\KeywordTok{FROM}\NormalTok{ detroit}
\end{Highlighting}
\end{Shaded}

\label{tab:unnamed-chunk-10}Displaying records 1 - 10

Year

FTP

UEMP

MAN

LIC

GR

CLEAR

WM

NMAN

GOV

HE

WE

HOM

ACC

ASR

1961

260.35

11.0

455.5

178.50

215.98

93.4

558724

538.1

133.9

2.98

117.18

8.60

39.17

306.18

1962

269.80

7.0

480.2

156.41

180.48

88.5

538584

547.6

137.6

3.09

134.02

8.90

40.27

315.16

1963

272.04

5.2

506.1

198.02

209.57

94.4

519171

562.8

143.6

3.23

141.68

8.52

45.31

277.53

1964

272.96

4.3

535.8

222.10

231.67

92.0

500457

591.0

150.3

3.33

147.98

8.89

49.51

234.07

1965

272.51

3.5

576.0

301.92

297.65

91.0

482418

626.1

164.3

3.46

159.85

13.07

55.05

230.84

1966

261.34

3.2

601.7

391.22

367.62

87.4

465029

659.8

179.5

3.60

157.19

14.57

53.90

217.99

1967

268.89

4.1

577.3

665.56

616.54

88.3

448267

686.2

187.5

3.73

155.29

21.36

50.62

286.11

1968

295.99

3.9

596.9

1131.21

1029.75

86.1

432109

699.6

195.4

2.91

131.75

28.03

51.47

291.59

1969

319.87

3.6

613.5

837.60

786.23

79.0

416533

729.9

210.3

4.25

178.74

31.49

49.16

320.39

1970

341.43

7.1

569.3

794.90

713.77

73.9

401518

757.8

223.8

4.47

178.30

37.39

45.80

323.03

\textbf{Student Dataset}

\begin{Shaded}
\begin{Highlighting}[]
\KeywordTok{SELECT}\NormalTok{ * }\KeywordTok{FROM}\NormalTok{ student}
\end{Highlighting}
\end{Shaded}

\label{tab:unnamed-chunk-11}Displaying records 1 - 10

school

sex

age

address

famsize

Pstatus

Medu

Fedu

Mjob

Fjob

reason

guardian

traveltime

studytime

failures

schoolsup

famsup

paid

activities

nursery

higher

internet

romantic

famrel

freetime

goout

Dalc

Walc

health

absences

G1

G2

G3

GP

F

18

U

GT3

A

4

4

at\_home

teacher

course

mother

2

2

0

yes

no

no

no

yes

yes

no

no

4

3

4

1

1

3

6

5

6

6

GP

F

17

U

GT3

T

1

1

at\_home

other

course

father

1

2

0

no

yes

no

no

no

yes

yes

no

5

3

3

1

1

3

4

5

5

6

GP

F

15

U

LE3

T

1

1

at\_home

other

other

mother

1

2

3

yes

no

yes

no

yes

yes

yes

no

4

3

2

2

3

3

10

7

8

10

GP

F

15

U

GT3

T

4

2

health

services

home

mother

1

3

0

no

yes

yes

yes

yes

yes

yes

yes

3

2

2

1

1

5

2

15

14

15

GP

F

16

U

GT3

T

3

3

other

other

home

father

1

2

0

no

yes

yes

no

yes

yes

no

no

4

3

2

1

2

5

4

6

10

10

GP

M

16

U

LE3

T

4

3

services

other

reputation

mother

1

2

0

no

yes

yes

yes

yes

yes

yes

no

5

4

2

1

2

5

10

15

15

15

GP

M

16

U

LE3

T

2

2

other

other

home

mother

1

2

0

no

no

no

no

yes

yes

yes

no

4

4

4

1

1

3

0

12

12

11

GP

F

17

U

GT3

A

4

4

other

teacher

home

mother

2

2

0

yes

yes

no

no

yes

yes

no

no

4

1

4

1

1

1

6

6

5

6

GP

M

15

U

LE3

A

3

2

services

other

home

mother

1

2

0

no

yes

yes

no

yes

yes

yes

no

4

2

2

1

1

1

0

16

18

19

GP

M

15

U

GT3

T

3

4

other

other

home

mother

1

2

0

no

yes

yes

yes

yes

yes

yes

no

5

5

1

1

1

5

0

14

15

15

Next we can list the tables that are available in our database

This is the DBI way to do it in Rstudio.

\begin{Shaded}
\begin{Highlighting}[]
\NormalTok{DBI}\OperatorTok{::}\KeywordTok{dbListTables}\NormalTok{(}\DataTypeTok{conn =}\NormalTok{ con)}
\end{Highlighting}
\end{Shaded}

\begin{verbatim}
## [1] "teachers" "dodgers"  "student"  "detroit"
\end{verbatim}

The following is the SQL syntax way to do it. Here we can see the
\emph{tablename,tableowner}, and the \emph{tablespace} along with other
housekeeping items that may be of interest to us.

\begin{Shaded}
\begin{Highlighting}[]
\KeywordTok{SELECT}\NormalTok{ * }\KeywordTok{FROM}\NormalTok{ pg_catalog.pg_tables;}
\end{Highlighting}
\end{Shaded}

\label{tab:unnamed-chunk-13}Displaying records 1 - 10

schemaname

tablename

tableowner

tablespace

hasindexes

hasrules

hastriggers

rowsecurity

public

teachers

ben

NA

FALSE

FALSE

FALSE

FALSE

pg\_catalog

pg\_statistic

postgres

NA

TRUE

FALSE

FALSE

FALSE

pg\_catalog

pg\_type

postgres

NA

TRUE

FALSE

FALSE

FALSE

public

dodgers

ben

NA

FALSE

FALSE

FALSE

FALSE

public

student

ben

NA

FALSE

FALSE

FALSE

FALSE

pg\_catalog

pg\_policy

postgres

NA

TRUE

FALSE

FALSE

FALSE

pg\_catalog

pg\_authid

postgres

pg\_global

TRUE

FALSE

FALSE

FALSE

public

detroit

ben

NA

FALSE

FALSE

FALSE

FALSE

pg\_catalog

pg\_user\_mapping

postgres

NA

TRUE

FALSE

FALSE

FALSE

pg\_catalog

pg\_subscription

postgres

pg\_global

TRUE

FALSE

FALSE

FALSE

\chapter{Select * FROM table;}\label{select-from-table}

Here we are querying the entire table and bringing back all of the
values.

\begin{verbatim}
## <PostgreSQLDriver>
\end{verbatim}

\begin{Shaded}
\begin{Highlighting}[]
\KeywordTok{SELECT}\NormalTok{ * }\KeywordTok{FROM}\NormalTok{ teachers}
\end{Highlighting}
\end{Shaded}

\label{tab:unnamed-chunk-15}6 records

id

first\_name

last\_name

school

hire\_date

salary

1

Janet

Smith

F.D. Roosevelt HS

2011-10-30

36200

2

Lee

Reynolds

F.D. Roosevelt HS

1993-05-22

65000

3

Samuel

Cole

Myers Middle School

2005-08-01

43500

4

Samantha

Bush

Myers Middle School

2011-10-30

36200

5

Betty

Diaz

Myers Middle School

2005-08-30

43500

6

Kathleen

Roush

F.D. Roosevelt HS

2010-10-22

38500

\begin{Shaded}
\begin{Highlighting}[]
\KeywordTok{SELECT}\NormalTok{ * }\KeywordTok{FROM}\NormalTok{ detroit}
\end{Highlighting}
\end{Shaded}

\label{tab:unnamed-chunk-16}Displaying records 1 - 10

Year

FTP

UEMP

MAN

LIC

GR

CLEAR

WM

NMAN

GOV

HE

WE

HOM

ACC

ASR

1961

260.35

11.0

455.5

178.50

215.98

93.4

558724

538.1

133.9

2.98

117.18

8.60

39.17

306.18

1962

269.80

7.0

480.2

156.41

180.48

88.5

538584

547.6

137.6

3.09

134.02

8.90

40.27

315.16

1963

272.04

5.2

506.1

198.02

209.57

94.4

519171

562.8

143.6

3.23

141.68

8.52

45.31

277.53

1964

272.96

4.3

535.8

222.10

231.67

92.0

500457

591.0

150.3

3.33

147.98

8.89

49.51

234.07

1965

272.51

3.5

576.0

301.92

297.65

91.0

482418

626.1

164.3

3.46

159.85

13.07

55.05

230.84

1966

261.34

3.2

601.7

391.22

367.62

87.4

465029

659.8

179.5

3.60

157.19

14.57

53.90

217.99

1967

268.89

4.1

577.3

665.56

616.54

88.3

448267

686.2

187.5

3.73

155.29

21.36

50.62

286.11

1968

295.99

3.9

596.9

1131.21

1029.75

86.1

432109

699.6

195.4

2.91

131.75

28.03

51.47

291.59

1969

319.87

3.6

613.5

837.60

786.23

79.0

416533

729.9

210.3

4.25

178.74

31.49

49.16

320.39

1970

341.43

7.1

569.3

794.90

713.77

73.9

401518

757.8

223.8

4.47

178.30

37.39

45.80

323.03

\begin{Shaded}
\begin{Highlighting}[]
\KeywordTok{SELECT}\NormalTok{ * }\KeywordTok{FROM}\NormalTok{ student}
\end{Highlighting}
\end{Shaded}

\label{tab:unnamed-chunk-17}Displaying records 1 - 10

school

sex

age

address

famsize

Pstatus

Medu

Fedu

Mjob

Fjob

reason

guardian

traveltime

studytime

failures

schoolsup

famsup

paid

activities

nursery

higher

internet

romantic

famrel

freetime

goout

Dalc

Walc

health

absences

G1

G2

G3

GP

F

18

U

GT3

A

4

4

at\_home

teacher

course

mother

2

2

0

yes

no

no

no

yes

yes

no

no

4

3

4

1

1

3

6

5

6

6

GP

F

17

U

GT3

T

1

1

at\_home

other

course

father

1

2

0

no

yes

no

no

no

yes

yes

no

5

3

3

1

1

3

4

5

5

6

GP

F

15

U

LE3

T

1

1

at\_home

other

other

mother

1

2

3

yes

no

yes

no

yes

yes

yes

no

4

3

2

2

3

3

10

7

8

10

GP

F

15

U

GT3

T

4

2

health

services

home

mother

1

3

0

no

yes

yes

yes

yes

yes

yes

yes

3

2

2

1

1

5

2

15

14

15

GP

F

16

U

GT3

T

3

3

other

other

home

father

1

2

0

no

yes

yes

no

yes

yes

no

no

4

3

2

1

2

5

4

6

10

10

GP

M

16

U

LE3

T

4

3

services

other

reputation

mother

1

2

0

no

yes

yes

yes

yes

yes

yes

no

5

4

2

1

2

5

10

15

15

15

GP

M

16

U

LE3

T

2

2

other

other

home

mother

1

2

0

no

no

no

no

yes

yes

yes

no

4

4

4

1

1

3

0

12

12

11

GP

F

17

U

GT3

A

4

4

other

teacher

home

mother

2

2

0

yes

yes

no

no

yes

yes

no

no

4

1

4

1

1

1

6

6

5

6

GP

M

15

U

LE3

A

3

2

services

other

home

mother

1

2

0

no

yes

yes

no

yes

yes

yes

no

4

2

2

1

1

1

0

16

18

19

GP

M

15

U

GT3

T

3

4

other

other

home

mother

1

2

0

no

yes

yes

yes

yes

yes

yes

no

5

5

1

1

1

5

0

14

15

15

\chapter{General Queries}\label{general-queries}

Here we will make a few queries to the database that are general in
nature. General queries are ones where we want to select particular
columns and also where we want to remove or delete items from the
database.

\section{Distinct Queries}\label{distinct-queries}

Here we will use distinct to look at the distinct values in a particular
column. This allows us to get a high-level overview of what our data
looks like.

\begin{Shaded}
\begin{Highlighting}[]
\KeywordTok{SELECT} \KeywordTok{distinct}\NormalTok{ school }\KeywordTok{from}\NormalTok{ teachers;}
\end{Highlighting}
\end{Shaded}

\label{tab:unnamed-chunk-18}2 records

school

Myers Middle School

F.D. Roosevelt HS

\begin{Shaded}
\begin{Highlighting}[]
\KeywordTok{select} \KeywordTok{distinct}\NormalTok{ school }\KeywordTok{from}\NormalTok{ student}
\end{Highlighting}
\end{Shaded}

\label{tab:unnamed-chunk-19}2 records

school

MS

GP

Next we can order our data in a particular way as well.

Notice that there is something very peculiar about this SQL statement.
If we write it out as a normal SQL statement it will not work.

\begin{Shaded}
\begin{Highlighting}[]
\NormalTok{RPostgreSQL}\OperatorTok{::}\KeywordTok{dbGetQuery}\NormalTok{(}\DataTypeTok{conn =}\NormalTok{ con,}\DataTypeTok{statement =} \StringTok{'select Medu from student;'}\NormalTok{)}
\end{Highlighting}
\end{Shaded}

\begin{verbatim}
## Error in postgresqlExecStatement(conn, statement, ...) : 
##   RS-DBI driver: (could not Retrieve the result : ERROR:  column "medu" does not exist
## LINE 1: select Medu from student;
##                ^
## HINT:  Perhaps you meant to reference the column "student.Medu" or the column "student.Fedu".
## )
\end{verbatim}

\begin{verbatim}
## Warning in postgresqlQuickSQL(conn, statement, ...): Could not create
## execute: select Medu from student;
\end{verbatim}

\begin{verbatim}
## NULL
\end{verbatim}

Instead we are required to utilize quotes around the column names since
it is Camelbacked: e.g.~Medu vs.~medu

\begin{Shaded}
\begin{Highlighting}[]
\NormalTok{RPostgreSQL}\OperatorTok{::}\KeywordTok{dbGetQuery}\NormalTok{(}\DataTypeTok{conn =}\NormalTok{ con, }\DataTypeTok{statement =} \StringTok{"select distinct }\CharTok{\textbackslash{}"}\StringTok{Medu}\CharTok{\textbackslash{}"}\StringTok{ from student;"}\NormalTok{)}
\end{Highlighting}
\end{Shaded}

\begin{verbatim}
##   Medu
## 1    0
## 2    1
## 3    3
## 4    2
## 5    4
\end{verbatim}

\begin{Shaded}
\begin{Highlighting}[]
\NormalTok{RPostgreSQL}\OperatorTok{::}\KeywordTok{dbGetQuery}\NormalTok{(}\DataTypeTok{conn =}\NormalTok{ con,}\DataTypeTok{statement =} \StringTok{'select "school","G3" from student order by "G3" desc limit 10;'}\NormalTok{)}
\end{Highlighting}
\end{Shaded}

\begin{verbatim}
##    school G3
## 1      GP 20
## 2      GP 19
## 3      GP 19
## 4      MS 19
## 5      GP 19
## 6      GP 19
## 7      GP 18
## 8      GP 18
## 9      GP 18
## 10     GP 18
\end{verbatim}

\section{WHERE Queries}\label{where-queries}

Where Clause in SQL

Here we are select the schools and the G3 grade where G3 is greater than
15.

\begin{Shaded}
\begin{Highlighting}[]
\NormalTok{RPostgreSQL}\OperatorTok{::}\KeywordTok{dbGetQuery}\NormalTok{(}\DataTypeTok{conn =}\NormalTok{ con, }\DataTypeTok{statement =} \StringTok{"select }\CharTok{\textbackslash{}"}\StringTok{school}\CharTok{\textbackslash{}"}\StringTok{,}\CharTok{\textbackslash{}"}\StringTok{G3}\CharTok{\textbackslash{}"}\StringTok{ from student where }\CharTok{\textbackslash{}"}\StringTok{G3}\CharTok{\textbackslash{}"}\StringTok{>15 limit 10;"}\NormalTok{)}
\end{Highlighting}
\end{Shaded}

\begin{verbatim}
##    school G3
## 1      GP 19
## 2      GP 16
## 3      GP 16
## 4      GP 17
## 5      GP 16
## 6      GP 18
## 7      GP 18
## 8      GP 20
## 9      GP 16
## 10     GP 16
\end{verbatim}

Here we are select the schools and the G3 grade where G3 is equal to 4.
Utilizing the operators \textgreater{},\textless{},= allows us to filter
our data and retrieve the data we want to look at.

\begin{Shaded}
\begin{Highlighting}[]
\NormalTok{RPostgreSQL}\OperatorTok{::}\KeywordTok{dbGetQuery}\NormalTok{(}\DataTypeTok{conn =}\NormalTok{ con,}\DataTypeTok{statement =} \StringTok{'select "school","G3" from student where "G3"=4;'}\NormalTok{)}
\end{Highlighting}
\end{Shaded}

\section{AND, OR, NOT Queries}\label{and-or-not-queries}

SQL Queries with \textbf{And, OR, NOT}.

\textbf{AND}

Here we only return one row where G3 = 4 and Medu = 4.

\begin{Shaded}
\begin{Highlighting}[]
\KeywordTok{select}\NormalTok{ * }\KeywordTok{from}\NormalTok{ student }\KeywordTok{where} \OtherTok{"G3"}\NormalTok{=}\DecValTok{4} \KeywordTok{AND} \OtherTok{"Medu"}\NormalTok{=}\DecValTok{4}\NormalTok{;}
\end{Highlighting}
\end{Shaded}

\label{tab:unnamed-chunk-25}1 records

school

sex

age

address

famsize

Pstatus

Medu

Fedu

Mjob

Fjob

reason

guardian

traveltime

studytime

failures

schoolsup

famsup

paid

activities

nursery

higher

internet

romantic

famrel

freetime

goout

Dalc

Walc

health

absences

G1

G2

G3

GP

F

17

U

GT3

T

4

3

other

other

reputation

mother

1

2

2

no

no

yes

no

yes

yes

yes

yes

3

4

5

2

4

1

22

6

6

4

school

sex

age

address

famsize

Pstatus

Medu

Fedu

Mjob

Fjob

reason

guardian

traveltime

studytime

failures

schoolsup

famsup

paid

activities

nursery

higher

internet

romantic

famrel

freetime

goout

Dalc

Walc

health

absences

G1

G2

G3

GP

F

17

U

GT3

T

4

3

other

other

reputation

mother

1

2

2

no

no

yes

no

yes

yes

yes

yes

3

4

5

2

4

1

22

6

6

4

\textbf{OR}

Here we return 10 rows where G3 = 4 or Mother's Education = 4. This
helps us to filter and sort data when we want to find something in
particular.

\begin{Shaded}
\begin{Highlighting}[]
\KeywordTok{select}\NormalTok{ * }\KeywordTok{from}\NormalTok{ student }\KeywordTok{where} \OtherTok{"G3"}\NormalTok{=}\DecValTok{4} \KeywordTok{OR} \OtherTok{"Medu"}\NormalTok{=}\DecValTok{4} \KeywordTok{LIMIT} \DecValTok{10}\NormalTok{;}
\end{Highlighting}
\end{Shaded}

\label{tab:unnamed-chunk-27}Displaying records 1 - 10

school

sex

age

address

famsize

Pstatus

Medu

Fedu

Mjob

Fjob

reason

guardian

traveltime

studytime

failures

schoolsup

famsup

paid

activities

nursery

higher

internet

romantic

famrel

freetime

goout

Dalc

Walc

health

absences

G1

G2

G3

GP

F

18

U

GT3

A

4

4

at\_home

teacher

course

mother

2

2

0

yes

no

no

no

yes

yes

no

no

4

3

4

1

1

3

6

5

6

6

GP

F

15

U

GT3

T

4

2

health

services

home

mother

1

3

0

no

yes

yes

yes

yes

yes

yes

yes

3

2

2

1

1

5

2

15

14

15

GP

M

16

U

LE3

T

4

3

services

other

reputation

mother

1

2

0

no

yes

yes

yes

yes

yes

yes

no

5

4

2

1

2

5

10

15

15

15

GP

F

17

U

GT3

A

4

4

other

teacher

home

mother

2

2

0

yes

yes

no

no

yes

yes

no

no

4

1

4

1

1

1

6

6

5

6

GP

F

15

U

GT3

T

4

4

teacher

health

reputation

mother

1

2

0

no

yes

yes

no

yes

yes

yes

no

3

3

3

1

2

2

0

10

8

9

GP

M

15

U

LE3

T

4

4

health

services

course

father

1

1

0

no

yes

yes

yes

yes

yes

yes

no

4

3

3

1

3

5

2

14

14

14

GP

M

15

U

GT3

T

4

3

teacher

other

course

mother

2

2

0

no

yes

yes

no

yes

yes

yes

no

5

4

3

1

2

3

2

10

10

11

GP

F

16

U

GT3

T

4

4

health

other

home

mother

1

1

0

no

yes

no

no

yes

yes

yes

no

4

4

4

1

2

2

4

14

14

14

GP

F

16

U

GT3

T

4

4

services

services

reputation

mother

1

3

0

no

yes

yes

yes

yes

yes

yes

no

3

2

3

1

2

2

6

13

14

14

GP

M

16

U

LE3

T

4

3

health

other

home

father

1

1

0

no

no

yes

yes

yes

yes

yes

no

3

1

3

1

3

5

4

8

10

10

\begin{Shaded}
\begin{Highlighting}[]
\NormalTok{RPostgreSQL}\OperatorTok{::}\KeywordTok{dbGetQuery}\NormalTok{(}\DataTypeTok{conn =}\NormalTok{ con,}\DataTypeTok{statement =} \StringTok{'select * from student where "G3"=4 OR "Medu"=4 LIMIT 10;'}\NormalTok{)}
\end{Highlighting}
\end{Shaded}

\begin{verbatim}
##    school sex age address famsize Pstatus Medu Fedu     Mjob     Fjob
## 1      GP   F  18       U     GT3       A    4    4  at_home  teacher
## 2      GP   F  15       U     GT3       T    4    2   health services
## 3      GP   M  16       U     LE3       T    4    3 services    other
## 4      GP   F  17       U     GT3       A    4    4    other  teacher
## 5      GP   F  15       U     GT3       T    4    4  teacher   health
## 6      GP   M  15       U     LE3       T    4    4   health services
## 7      GP   M  15       U     GT3       T    4    3  teacher    other
## 8      GP   F  16       U     GT3       T    4    4   health    other
## 9      GP   F  16       U     GT3       T    4    4 services services
## 10     GP   M  16       U     LE3       T    4    3   health    other
##        reason guardian traveltime studytime failures schoolsup famsup paid
## 1      course   mother          2         2        0       yes     no   no
## 2        home   mother          1         3        0        no    yes  yes
## 3  reputation   mother          1         2        0        no    yes  yes
## 4        home   mother          2         2        0       yes    yes   no
## 5  reputation   mother          1         2        0        no    yes  yes
## 6      course   father          1         1        0        no    yes  yes
## 7      course   mother          2         2        0        no    yes  yes
## 8        home   mother          1         1        0        no    yes   no
## 9  reputation   mother          1         3        0        no    yes  yes
## 10       home   father          1         1        0        no     no  yes
##    activities nursery higher internet romantic famrel freetime goout Dalc
## 1          no     yes    yes       no       no      4        3     4    1
## 2         yes     yes    yes      yes      yes      3        2     2    1
## 3         yes     yes    yes      yes       no      5        4     2    1
## 4          no     yes    yes       no       no      4        1     4    1
## 5          no     yes    yes      yes       no      3        3     3    1
## 6         yes     yes    yes      yes       no      4        3     3    1
## 7          no     yes    yes      yes       no      5        4     3    1
## 8          no     yes    yes      yes       no      4        4     4    1
## 9         yes     yes    yes      yes       no      3        2     3    1
## 10        yes     yes    yes      yes       no      3        1     3    1
##    Walc health absences G1 G2 G3
## 1     1      3        6  5  6  6
## 2     1      5        2 15 14 15
## 3     2      5       10 15 15 15
## 4     1      1        6  6  5  6
## 5     2      2        0 10  8  9
## 6     3      5        2 14 14 14
## 7     2      3        2 10 10 11
## 8     2      2        4 14 14 14
## 9     2      2        6 13 14 14
## 10    3      5        4  8 10 10
\end{verbatim}

\textbf{NOT}

Here we are looking at results where Fathers Education (Fedu) does not
equal 4.

\begin{Shaded}
\begin{Highlighting}[]
\NormalTok{RPostgreSQL}\OperatorTok{::}\KeywordTok{dbGetQuery}\NormalTok{(}\DataTypeTok{conn =}\NormalTok{ con,}\DataTypeTok{statement =} \StringTok{'select * from student WHERE NOT "Fedu"=4 LIMIT 10;'}\NormalTok{)}
\end{Highlighting}
\end{Shaded}

\begin{Shaded}
\begin{Highlighting}[]
\KeywordTok{select}\NormalTok{ * }\KeywordTok{from}\NormalTok{ student }\KeywordTok{WHERE} \KeywordTok{NOT} \OtherTok{"Fedu"}\NormalTok{=}\DecValTok{4} \KeywordTok{LIMIT} \DecValTok{10}\NormalTok{;}
\end{Highlighting}
\end{Shaded}

\label{tab:unnamed-chunk-30}Displaying records 1 - 10

school

sex

age

address

famsize

Pstatus

Medu

Fedu

Mjob

Fjob

reason

guardian

traveltime

studytime

failures

schoolsup

famsup

paid

activities

nursery

higher

internet

romantic

famrel

freetime

goout

Dalc

Walc

health

absences

G1

G2

G3

GP

F

17

U

GT3

T

1

1

at\_home

other

course

father

1

2

0

no

yes

no

no

no

yes

yes

no

5

3

3

1

1

3

4

5

5

6

GP

F

15

U

LE3

T

1

1

at\_home

other

other

mother

1

2

3

yes

no

yes

no

yes

yes

yes

no

4

3

2

2

3

3

10

7

8

10

GP

F

15

U

GT3

T

4

2

health

services

home

mother

1

3

0

no

yes

yes

yes

yes

yes

yes

yes

3

2

2

1

1

5

2

15

14

15

GP

F

16

U

GT3

T

3

3

other

other

home

father

1

2

0

no

yes

yes

no

yes

yes

no

no

4

3

2

1

2

5

4

6

10

10

GP

M

16

U

LE3

T

4

3

services

other

reputation

mother

1

2

0

no

yes

yes

yes

yes

yes

yes

no

5

4

2

1

2

5

10

15

15

15

GP

M

16

U

LE3

T

2

2

other

other

home

mother

1

2

0

no

no

no

no

yes

yes

yes

no

4

4

4

1

1

3

0

12

12

11

GP

M

15

U

LE3

A

3

2

services

other

home

mother

1

2

0

no

yes

yes

no

yes

yes

yes

no

4

2

2

1

1

1

0

16

18

19

GP

F

15

U

GT3

T

2

1

services

other

reputation

father

3

3

0

no

yes

no

yes

yes

yes

yes

no

5

2

2

1

1

4

4

10

12

12

GP

M

15

U

GT3

T

4

3

teacher

other

course

mother

2

2

0

no

yes

yes

no

yes

yes

yes

no

5

4

3

1

2

3

2

10

10

11

GP

M

15

U

GT3

A

2

2

other

other

home

other

1

3

0

no

yes

no

no

yes

yes

yes

yes

4

5

2

1

1

3

0

14

16

16

\textbf{Combining AND, OR, NOT}

\begin{Shaded}
\begin{Highlighting}[]
\NormalTok{RPostgreSQL}\OperatorTok{::}\KeywordTok{dbGetQuery}\NormalTok{(}\DataTypeTok{conn =}\NormalTok{ con,}\DataTypeTok{statement =} \StringTok{'select * from student where "G3">=10 OR "Medu"=4 LIMIT 10;'}\NormalTok{)}
\end{Highlighting}
\end{Shaded}

\begin{Shaded}
\begin{Highlighting}[]
\KeywordTok{select}\NormalTok{ * }\KeywordTok{from}\NormalTok{ student }\KeywordTok{where} \OtherTok{"G3"}\NormalTok{>=}\DecValTok{10} \KeywordTok{OR} \OtherTok{"Medu"}\NormalTok{=}\DecValTok{4} \KeywordTok{LIMIT} \DecValTok{10}\NormalTok{;}
\end{Highlighting}
\end{Shaded}

\label{tab:unnamed-chunk-32}Displaying records 1 - 10

school

sex

age

address

famsize

Pstatus

Medu

Fedu

Mjob

Fjob

reason

guardian

traveltime

studytime

failures

schoolsup

famsup

paid

activities

nursery

higher

internet

romantic

famrel

freetime

goout

Dalc

Walc

health

absences

G1

G2

G3

GP

F

18

U

GT3

A

4

4

at\_home

teacher

course

mother

2

2

0

yes

no

no

no

yes

yes

no

no

4

3

4

1

1

3

6

5

6

6

GP

F

15

U

LE3

T

1

1

at\_home

other

other

mother

1

2

3

yes

no

yes

no

yes

yes

yes

no

4

3

2

2

3

3

10

7

8

10

GP

F

15

U

GT3

T

4

2

health

services

home

mother

1

3

0

no

yes

yes

yes

yes

yes

yes

yes

3

2

2

1

1

5

2

15

14

15

GP

F

16

U

GT3

T

3

3

other

other

home

father

1

2

0

no

yes

yes

no

yes

yes

no

no

4

3

2

1

2

5

4

6

10

10

GP

M

16

U

LE3

T

4

3

services

other

reputation

mother

1

2

0

no

yes

yes

yes

yes

yes

yes

no

5

4

2

1

2

5

10

15

15

15

GP

M

16

U

LE3

T

2

2

other

other

home

mother

1

2

0

no

no

no

no

yes

yes

yes

no

4

4

4

1

1

3

0

12

12

11

GP

F

17

U

GT3

A

4

4

other

teacher

home

mother

2

2

0

yes

yes

no

no

yes

yes

no

no

4

1

4

1

1

1

6

6

5

6

GP

M

15

U

LE3

A

3

2

services

other

home

mother

1

2

0

no

yes

yes

no

yes

yes

yes

no

4

2

2

1

1

1

0

16

18

19

GP

M

15

U

GT3

T

3

4

other

other

home

mother

1

2

0

no

yes

yes

yes

yes

yes

yes

no

5

5

1

1

1

5

0

14

15

15

GP

F

15

U

GT3

T

4

4

teacher

health

reputation

mother

1

2

0

no

yes

yes

no

yes

yes

yes

no

3

3

3

1

2

2

0

10

8

9

\textbf{The AND OR}

Here we tell SQL that we want all the G3 grades that are \textgreater{}
10 and also that the school should be GP OR Fedu should equal 4. The
backslashes allow us to escape the single quotes that are necessary when
using RPostgresql syntax.

\begin{Shaded}
\begin{Highlighting}[]
\NormalTok{RPostgreSQL}\OperatorTok{::}\KeywordTok{dbGetQuery}\NormalTok{(}\DataTypeTok{conn =}\NormalTok{ con,}\DataTypeTok{statement =} \StringTok{'select * from student where "G3">=10 AND school=}\CharTok{\textbackslash{}'}\StringTok{GP}\CharTok{\textbackslash{}'}\StringTok{ OR "Fedu"=4 LIMIT 10;'}\NormalTok{)}
\end{Highlighting}
\end{Shaded}

\begin{Shaded}
\begin{Highlighting}[]
\KeywordTok{select} \OtherTok{"G3"}\NormalTok{, school, }\OtherTok{"Fedu"} \KeywordTok{from}\NormalTok{ student }\KeywordTok{where} \OtherTok{"G3"}\NormalTok{>=}\DecValTok{10} \KeywordTok{AND}\NormalTok{ school=}\StringTok{'GP'} \KeywordTok{OR} \OtherTok{"Fedu"}\NormalTok{=}\DecValTok{4} \KeywordTok{LIMIT} \DecValTok{10}\NormalTok{;}
\end{Highlighting}
\end{Shaded}

\label{tab:unnamed-chunk-34}Displaying records 1 - 10

G3

school

Fedu

6

GP

4

10

GP

1

15

GP

2

10

GP

3

15

GP

3

11

GP

2

6

GP

4

19

GP

2

15

GP

4

9

GP

4

\textbf{The double NOT or NOT NOT}

Here we tell SQL that we want to return all values where Fedu and Medu
are not equal to 4.

\begin{Shaded}
\begin{Highlighting}[]
\NormalTok{RPostgreSQL}\OperatorTok{::}\KeywordTok{dbGetQuery}\NormalTok{(}\DataTypeTok{conn =}\NormalTok{ con,}\DataTypeTok{statement =} \StringTok{'select * from student WHERE NOT "Medu"=4 OR "Fedu"=4 LIMIT 10;'}\NormalTok{)}
\end{Highlighting}
\end{Shaded}

\begin{verbatim}
##    school sex age address famsize Pstatus Medu Fedu     Mjob    Fjob
## 1      GP   F  18       U     GT3       A    4    4  at_home teacher
## 2      GP   F  17       U     GT3       T    1    1  at_home   other
## 3      GP   F  15       U     LE3       T    1    1  at_home   other
## 4      GP   F  16       U     GT3       T    3    3    other   other
## 5      GP   M  16       U     LE3       T    2    2    other   other
## 6      GP   F  17       U     GT3       A    4    4    other teacher
## 7      GP   M  15       U     LE3       A    3    2 services   other
## 8      GP   M  15       U     GT3       T    3    4    other   other
## 9      GP   F  15       U     GT3       T    4    4  teacher  health
## 10     GP   F  15       U     GT3       T    2    1 services   other
##        reason guardian traveltime studytime failures schoolsup famsup paid
## 1      course   mother          2         2        0       yes     no   no
## 2      course   father          1         2        0        no    yes   no
## 3       other   mother          1         2        3       yes     no  yes
## 4        home   father          1         2        0        no    yes  yes
## 5        home   mother          1         2        0        no     no   no
## 6        home   mother          2         2        0       yes    yes   no
## 7        home   mother          1         2        0        no    yes  yes
## 8        home   mother          1         2        0        no    yes  yes
## 9  reputation   mother          1         2        0        no    yes  yes
## 10 reputation   father          3         3        0        no    yes   no
##    activities nursery higher internet romantic famrel freetime goout Dalc
## 1          no     yes    yes       no       no      4        3     4    1
## 2          no      no    yes      yes       no      5        3     3    1
## 3          no     yes    yes      yes       no      4        3     2    2
## 4          no     yes    yes       no       no      4        3     2    1
## 5          no     yes    yes      yes       no      4        4     4    1
## 6          no     yes    yes       no       no      4        1     4    1
## 7          no     yes    yes      yes       no      4        2     2    1
## 8         yes     yes    yes      yes       no      5        5     1    1
## 9          no     yes    yes      yes       no      3        3     3    1
## 10        yes     yes    yes      yes       no      5        2     2    1
##    Walc health absences G1 G2 G3
## 1     1      3        6  5  6  6
## 2     1      3        4  5  5  6
## 3     3      3       10  7  8 10
## 4     2      5        4  6 10 10
## 5     1      3        0 12 12 11
## 6     1      1        6  6  5  6
## 7     1      1        0 16 18 19
## 8     1      5        0 14 15 15
## 9     2      2        0 10  8  9
## 10    1      4        4 10 12 12
\end{verbatim}

\section{Insert Queries}\label{insert-queries}

Insert into PostgreSQL using RPostgreSQL

\begin{itemize}
\tightlist
\item
  Inserting a single list of values into PostgreSQL.
\end{itemize}

Ok, now lets INSERT some data into our PostgreSQL database. We will want
to develop a query string and send this to the database via
dbSendQuery() from the RPostgreSQL package.

\begin{Shaded}
\begin{Highlighting}[]
\NormalTok{query <-}\StringTok{ }\NormalTok{(}\StringTok{'INSERT INTO detroit VALUES (265, 14, 500.5, 200.5, 215.98, 93.457, 558724, 538.123, 133.96, 2.75, 117.187, 8.564, 39.17, 306.18);'}\NormalTok{)}

\NormalTok{table<-}\StringTok{ }\NormalTok{RPostgreSQL}\OperatorTok{::}\KeywordTok{dbSendQuery}\NormalTok{(}\DataTypeTok{conn =}\NormalTok{ con,}\DataTypeTok{statement =}\NormalTok{ query)}
\end{Highlighting}
\end{Shaded}

Then when we call the new data we can see that we have updated the row
(observations) to 14 and have added the data in our query.

\begin{Shaded}
\begin{Highlighting}[]
\NormalTok{RPostgreSQL}\OperatorTok{::}\KeywordTok{dbGetQuery}\NormalTok{(}\DataTypeTok{conn =}\NormalTok{ con,}\DataTypeTok{statement =} \StringTok{'select * from detroit;'}\NormalTok{)}
\end{Highlighting}
\end{Shaded}

\begin{verbatim}
##    Year    FTP  UEMP   MAN     LIC       GR    CLEAR     WM   NMAN    GOV
## 1  1961 260.35  11.0 455.5  178.50  215.980     93.4 558724 538.10 133.90
## 2  1962 269.80   7.0 480.2  156.41  180.480     88.5 538584 547.60 137.60
## 3  1963 272.04   5.2 506.1  198.02  209.570     94.4 519171 562.80 143.60
## 4  1964 272.96   4.3 535.8  222.10  231.670     92.0 500457 591.00 150.30
## 5  1965 272.51   3.5 576.0  301.92  297.650     91.0 482418 626.10 164.30
## 6  1966 261.34   3.2 601.7  391.22  367.620     87.4 465029 659.80 179.50
## 7  1967 268.89   4.1 577.3  665.56  616.540     88.3 448267 686.20 187.50
## 8  1968 295.99   3.9 596.9 1131.21 1029.750     86.1 432109 699.60 195.40
## 9  1969 319.87   3.6 613.5  837.60  786.230     79.0 416533 729.90 210.30
## 10 1970 341.43   7.1 569.3  794.90  713.770     73.9 401518 757.80 223.80
## 11 1971 356.59   8.4 548.8  817.74  750.430     63.4 387046 755.30 227.70
## 12 1972 376.69   7.7 563.4  583.17 1027.380     62.5 373095 787.00 230.90
## 13 1973 390.19   6.3 609.3  709.59  666.500     58.9 359647 819.80 230.20
## 14  265  14.00 500.5 200.5  215.98   93.457 558724.0    538 133.96   2.75
##         HE      WE   HOM    ACC    ASR
## 1    2.980 117.180  8.60  39.17 306.18
## 2    3.090 134.020  8.90  40.27 315.16
## 3    3.230 141.680  8.52  45.31 277.53
## 4    3.330 147.980  8.89  49.51 234.07
## 5    3.460 159.850 13.07  55.05 230.84
## 6    3.600 157.190 14.57  53.90 217.99
## 7    3.730 155.290 21.36  50.62 286.11
## 8    2.910 131.750 28.03  51.47 291.59
## 9    4.250 178.740 31.49  49.16 320.39
## 10   4.470 178.300 37.39  45.80 323.03
## 11   5.040 209.540 46.26  44.54 357.38
## 12   5.470 240.050 47.24  41.03 422.07
## 13   5.760 258.050 52.33  44.17 473.01
## 14 117.187   8.564 39.17 306.18     NA
\end{verbatim}

\chapter{RPostgreSQL in Shiny
Applications}\label{rpostgresql-in-shiny-applications}

Here we will show you how to use RPostgreSQL within your R-Shiny or
Shiny application. This can be somewhat frustrating as you will need to
take advantage of the paste() and paste0() base commands in R to send
your text or numeric input data to the query itself.

\begin{Shaded}
\begin{Highlighting}[]
\KeywordTok{paste}\NormalTok{(}\DataTypeTok{x =}\NormalTok{ name ,}\DataTypeTok{sep =} \StringTok{"-"}\NormalTok{,}\DataTypeTok{collapse =} \StringTok{""}\NormalTok{)}

\KeywordTok{paste0}\NormalTok{()}
\end{Highlighting}
\end{Shaded}

\section{Insert Query from Shiny
Application}\label{insert-query-from-shiny-application}

For us to send a query to the PostgreSQL database we will need to paste
the query together. We will begin by constructing the usual
\textbf{INSERT INTO TABLENAME} and then insert the respective values
into the query as well. We will need to utilize the
\textbf{input\(value1** replacing whatever **input\)name} we have for
the value we want to insert into the databale when the query is sent. We
can simply paste this into a new variable called \textbf{qry} and assign
it and pass this \textbf{qry} to the \textbf{dbSendQuery} function in
our application. As the number of columns increase so will the number of
values as well.

\begin{Shaded}
\begin{Highlighting}[]
\NormalTok{      qry =}\StringTok{ }\KeywordTok{paste0}\NormalTok{(}\StringTok{"INSERT INTO table (column1,column2)"}\NormalTok{,}
                   \StringTok{"VALUES ('"}\NormalTok{,}\KeywordTok{paste}\NormalTok{(input}\OperatorTok{$}\NormalTok{value1,}\StringTok{"'"}\NormalTok{,}\StringTok{","}\NormalTok{,}\StringTok{"'"}\NormalTok{,input}\OperatorTok{$}\NormalTok{value2,}\StringTok{"')"}\NormalTok{))}

    \KeywordTok{dbSendQuery}\NormalTok{(}\DataTypeTok{conn =}\NormalTok{ con, }\DataTypeTok{statement =}\NormalTok{ qry)}
\end{Highlighting}
\end{Shaded}

\textbf{Note: We have each value surrounded by the single back ticks '
and also we have the values seperated by a comma as well.}

\section{Write Table Query from Shiny Application to the
Database}\label{write-table-query-from-shiny-application-to-the-database}

Here you can take a dataframe and write it directly to the database
using the \textbf{dbWriteTable} command. This allows us to write a
dataframe directlyt to the table in question. You will need the
dataframe column names to match the ones that are in the table in the
database. The number of columns must match as well. You will not be able
to write a dataframe with more columns than are in the table, but you
will be able to write a dataframe that has less columns than are in the
table in the database.

\begin{Shaded}
\begin{Highlighting}[]
\KeywordTok{dbWriteTable}\NormalTok{(}\DataTypeTok{conn =}\NormalTok{ con,}\DataTypeTok{name =} \StringTok{'table_name'}\NormalTok{,}\DataTypeTok{value =}\NormalTok{ table_value)}

\NormalTok{RPostgreSQL}\OperatorTok{::}\KeywordTok{postgresqlWriteTable}\NormalTok{(}\DataTypeTok{con =}\NormalTok{ con,}\DataTypeTok{name =} \StringTok{'table_name'}\NormalTok{,}\DataTypeTok{value =}\NormalTok{ table_value,}\DataTypeTok{overwrite =} \OtherTok{TRUE}\NormalTok{)}

\NormalTok{RPostgreSQL}\OperatorTok{::}\KeywordTok{postgresqlWriteTable}\NormalTok{(}\DataTypeTok{con =}\NormalTok{ con,}\DataTypeTok{name =} \StringTok{'table_name'}\NormalTok{,}\DataTypeTok{value =}\NormalTok{ table_value,}\DataTypeTok{append =} \OtherTok{TRUE}\NormalTok{)}
\end{Highlighting}
\end{Shaded}

Above we have 3 distinct ways to write the data to the table in the
database. The last two allow us to either \textbf{overwrite} or
\textbf{append} the data to the table. Depending on our application
needs we will be able to do one or the other.


\end{document}
